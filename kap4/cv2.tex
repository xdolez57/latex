\documentclass[a4paper]{article}
\usepackage[czech]{babel}
\usepackage[T1]{fontenc}
\usepackage[utf8]{inputenc}

\begin{document}

Pojmy týkající se horského kola:
\begin{description}

\item [rám] -- dnes nejčastěji hliníkový (duralový), méně často velmi drahý karbonový, popř. u starých kol těžký ocelový. Může mít pevnou zadní stavbu (tzv. hardtail) anebo se může jednat o celoodpružený rám, který má v zadní stavbě zabudovaný tlumič (pružení).
\item [kola] -- jedno přední a jedno zadní o průměru 26, 27,5 nebo 29 palců. Každé kolo se skládá z ráfku, výpletu a náboje. Nejčastějším servisním úkonem, spojeným s koly je centrování.
\item [vidlice] -- pevná nebo odpružená. Odpružené vidlice se liší v mnoha parametrech a použitých technologiích a tedy i v ceně.
\item [přehazovačka] -- nejznámějším výrobcem přehazovaček je Shimano, následován značkou Sram. Důležitým parametrem je počet převodů a hmotnost.
\item [kazeta] -- liší se množstvím převodů (pastorků) a jejich velikostí -- počtem zubů na jednotlivých pastorcích. Počet převodů musí souhlasit s modelem použité přehazovačky.
\item [převodníky] -- Podobně jako u kazety je důležitým parametrem počet zubů na jednotlivých převodnících. To má významný vliv na celkový převodový poměr. Stále nejčastější kola mají tři převodníky, ale stále oblíbenější jsou kola se dvěmi, popř. pouhým jedním převodníkem. Počet převodníků opět musí souhlasit s typem přesmykače. Největší výhodou jednopřevodníku je absence přesmykače a tedy také snížení hmotnosti kola. Zároveň odpadá nutnost seřizování přesmykače. Nevýhodou je naopak menší celkové množství převodů a často také menší rozsah převodů.
\item [pláště] -- Na trhu existuje velké množství různých značek a typů plášťů. Ty se liší tvarem vzorku, velikostí, materiálem, ze kterého jsou vyrobeny. Pro každý povrch a typ kola jsou vhodné jiné pláště. Zajímavou variantou jsou bezdušové pláště.
\item [pedály] -- Běžné pedály mohou být buď z hliníku nebo z plastu. Pro náročnější cyklisty je možno sáhnout po tzv. nášlapných pedálech, které fixují nohu na pedál a je tedy možno při šlapání tlačit nejenom směrem dolů, ale také táhnout pedál směrem nahoru. Nevýhodou je vyšší cena a také nutnost současně pořídit specializovanou obuv -- tzv. cyklistické tretry.
\item [řetěz] -- Jedná se o spotřební komponentu kola. Pomocí speciální měrky je možno změřit vytahání řetězu a při případném nadměrném opotřebení je nutné řetěz vyměnit. Pokud není nadměrně opotřebený řetěz vyměněn, může dojít ke zničení kazety a převodníků. Řetěz je také nutné pravidelně čistit a mazat vhodným mazivem.
\item [příslušenství] -- V dnešní době existuje nepřeberné množství příslušenství pro cyklisty. Základně vybavené kolo by mělo mít dostatečné množství odrazek a osvětlení kola pro jízdu za snížené viditelnosti. Dalšími možnostmi jsou různé nosiče, brašny, cyklistické computery atd.

\end{description}

\end{document}
